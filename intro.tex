\documentclass[ notitlepage, numerical, 11pt]{revtex4-1} % style for Physical Review B and AJP are similar
\usepackage{float}
\usepackage{amsmath}

\usepackage{graphicx} 


\usepackage{sverb}

\begin{document}

\title{Physics of Non-Volatile Digital Data Storage Technologies}
\author{Tenzin Rigden}
\affiliation{Carleton College, Department of Physics, Northfield, MN 55057}

\begin{abstract}
Data storage technologies have become increasingly important as more and more of our lives are being stored digitally in our phones and computers, and the need for higher density and faster data storage technologies increases. In this paper, I will talk about the various technologies used for non-volatile data storage, non-volatile meaning that do not lose their data when not powered. I will begin with modern technologies such as optical disks, hard disk drives, and flash storage that are currently being used. In addition, I will talk about currently in development technologies such as holographic data storage and probe based storage that promise much higher storage densities than currently offered. 
\end{abstract}

\maketitle
\section{Introduction}
Computers and phones have become such a ubiquitous part of our lives that it is becoming more and more difficult to live without them. However, just as important as these devices themselves is what is stored in them. Your pictures, music, videos, papers, and more are all stored digitally on the devices. As more and more of our lives are recorded digitally, it has become increasingly important to find new ways to either expand our current data storage technologies or find new ones. On a fundamental level, a computer stores everything in a series of 1s or 0s called bits. In this binary system, 1 represents a ''true" state while a 0 represents a ''false" state. By using this binary system, it is possible to represent numbers as a sequence of 1s and 0s where each digit starting from the right represent an increasing power of 2 starting with $2^0$. For example, the number 13 can be represented as 1101 so we get, from the right, 1*$2^0$ + 0*$2^1$ + 1*$2^2$ + 1*$2^3$ which adds up to 13. Now that we can represent numbers, we can use these numbers to represent other characters. One method is to use the American Standard Code for Information Interchange (ASCII) encoding system which was first published in 1963. ASCII encodes 128 characters that include letters, both lower case and capital, numbers, and other special characters into 7 bit integers. To represent the capital letter T, that corresponds to the 84th character and thus in binary can be represented as 01010100. A series of these binary numbers can be used to represent text.


Now that we know we can use binary to represent text, we can look at how we can physically write these 1s and 0s. One of the earlier methods was to use punch cards which used holes and not holes to represent 1s and 0s. The early punch cards used 36 bits words because they were used in calculators and they wanted to be able to represent 10 decimal places. However this method is not very dense, data storage wise.


In this paper, I will talk about more modern techniques that are currently being used for non-volatile memory, which means that data will not be lost if it does not have power. I will begin with optical data storage, specifically optical disks. Then I will talk about magnetic data storage in talking about hard disk drives and their advancements. Then I will talk about flash memory storage which promises much faster access to data. Lastly, I will talk about state of the art techniques currently being developed such as holographic data storage and probe based storage that could greatly increase storage density.





\end{document}
